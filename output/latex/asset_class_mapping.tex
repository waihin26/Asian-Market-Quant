\documentclass{article}
\usepackage[margin=1in]{geometry}
\usepackage{tabularx}
\usepackage{booktabs}
\usepackage{color}
\usepackage{graphicx}

\title{Asset Class Mapping for Asian Market Quant Project}
\author{Wong Wai Hin}
\date{\today}

\begin{document}

\maketitle

\section{Asset Class Mapping}

This document outlines the asset class categorization for our cross-asset Asian markets project. 
We have categorized the tickers into five main asset classes for systematic analysis and risk budgeting.


\begin{table}[h!]
\centering
\caption{Asset Class Mapping for Asian Markets}
\label{tab:asset_class_mapping}
\begin{tabularx}{\textwidth}{|l|X|l|X|}
\hline
\textbf{Ticker Range} & \textbf{Asset Class} & \textbf{Currency} & \textbf{Comment} \\
\hline
MXAP \ldots FMETF & Emerging-Asia equity indices & Mostly USD & Regional beta + macro sensitivity \\
\hline
GOLDS \ldots S & Commodities (Gold spot, Brent front-month, generic Softs, Philippines gold ETF) & USD & Adds inflation hedge, carry via roll \\
\hline
SPX \ldots NKY & Developed-market equity benchmarks & USD / JPY & Good stress-test proxies \\
\hline
USDPHP \ldots USDJPY & FX & USD notional & Carry + momentum rich \\
\hline
USGG5YR \ldots GTUSDPH5Y & Sovereign & USD & Duration + EM credit risk \\
\hline
\end{tabularx}
\end{table}

\section{Risk Budgeting Framework}

Based on our asset class mapping, we propose the following risk budget allocation for portfolio construction.
This allocation will be used in the hierarchical risk parity (HRP) overlay.


\begin{table}[h!]
\centering
\caption{Risk Budget Allocation}
\label{tab:risk_budget}
\begin{tabular}{|l|c|}
\hline
\textbf{Asset Class} & \textbf{Allocation (\%)} \\
\hline
Equities & 60.0\% \\
\hline
Rates & 20.0\% \\
\hline
Fx & 10.0\% \\
\hline
Commodities & 10.0\% \\
\hline
\end{tabular}
\end{table}

\section{Asset Class Descriptions}

\subsection{Emerging-Asia Equity (12 instruments)}
\textbf{Description:} Emerging-Asia equity indices \& ETFs provide broad exposure to regional equity markets.\\
\begin{itemize}
\item \textbf{Currency:} Mostly USD
\item \textbf{Strategy Rationale:} Regional beta + macro sensitivity, liquid access to Asian growth
\item \textbf{Instruments:}
  \begin{itemize}
    \item MXAP Index - MSCI Asia Pacific Index
    \item MXAPJ Index - MSCI Asia Pacific ex-Japan Index
    \item MXAS Index - MSCI Asia Index
    \item MXASJ Index - MSCI Asia ex-Japan Index
    \item PCOMP Index - Philippines Stock Exchange Index
    \item JCI Index - Jakarta Composite Index
    \item FBMKLCI Index - FTSE Bursa Malaysia KLCI
    \item SET Index - Stock Exchange of Thailand Index
    \item STI Index - Straits Times Index
    \item NU710465 Index - Asia Region Index
    \item EPHE US Index - iShares MSCI Philippines ETF
    \item FMETF PM Equity - First Metro Philippine Equity ETF
  \end{itemize}
\end{itemize}

\subsection{Commodities (3 instruments)}
\textbf{Description:} Strategic commodity exposure through major benchmarks\\
\begin{itemize}
\item \textbf{Currency:} USD
\item \textbf{Strategy Rationale:} Inflation hedge + carry opportunities via futures roll
\item \textbf{Instruments:}
  \begin{itemize}
    \item GOLDS Index - Gold Spot Price
    \item CO1 Comdty - Brent Crude Oil Front Month
    \item S1 Comdty - Generic Softs Contract
  \end{itemize}
\end{itemize}

\subsection{Developed Equity (2 instruments)}
\textbf{Description:} Major developed market benchmarks\\
\begin{itemize}
\item \textbf{Currency:} USD/JPY
\item \textbf{Strategy Rationale:} Global risk sentiment indicators + regime detection
\item \textbf{Instruments:}
  \begin{itemize}
    \item SPX Index - S\&P 500 Index
    \item NKY Index - Nikkei 225
  \end{itemize}
\end{itemize}

\subsection{FX Crosses (5 instruments)}
\textbf{Description:} Major Asian currency pairs vs USD\\
\begin{itemize}
\item \textbf{Currency:} USD crosses
\item \textbf{Strategy Rationale:} Carry opportunities + momentum signals
\item \textbf{Instruments:}
  \begin{itemize}
    \item USDPHP Index - USD/PHP Exchange Rate
    \item USDMYR Index - USD/MYR Exchange Rate
    \item USDIDR Index - USD/IDR Exchange Rate
    \item USDSGD Index - USD/SGD Exchange Rate
    \item USDJPY Index - USD/JPY Exchange Rate
  \end{itemize}
\end{itemize}

\subsection{Sovereign Yields (3 instruments)}
\textbf{Description:} Government bonds and sovereign debt\\
\begin{itemize}
\item \textbf{Currency:} USD and local currency
\item \textbf{Strategy Rationale:} Duration exposure + EM credit risk premium
\item \textbf{Instruments:}
  \begin{itemize}
    \item USGG5YR Index - US 5-Year Treasury Yield
    \item GTPHP5yr Corp - Philippines 5-Year Government Bond
    \item GTUSDPH5Y Corp - USD-denominated Philippines 5-Year Sovereign Bond
  \end{itemize}
\end{itemize}

\section{Portfolio Construction Process}

\subsection{Risk Budgeting Implementation}
The portfolio construction process follows a hierarchical approach:
\begin{enumerate}
    \item \textbf{Asset Class Level (Stage 1)}
    \begin{itemize}
        \item Apply risk budget weights to each asset class
        \item Emerging Asia Equity (35\%) - Regional growth exposure
        \item Developed Equity (25\%) - Global risk factors
        \item Sovereign Yields (20\%) - Duration + credit exposure
        \item FX (10\%) - Carry + momentum factors
        \item Commodities (10\%) - Inflation hedge
    \end{itemize}
    
    \item \textbf{Instrument Level (Stage 2)}
    \begin{itemize}
        \item Within each asset class, apply hierarchical risk parity
        \item Account for correlation clusters
        \item Adjust for liquidity constraints
        \item Consider market microstructure
    \end{itemize}
\end{enumerate}

\section{Implementation Roadmap}

\begin{enumerate}
    \item \textbf{Data Engineering (Current Phase)}
    \begin{itemize}
        \item Implement robust data cleaning pipeline
        \item Normalize all prices to USD
        \item Handle corporate actions and futures rolls
        \item Create standardized return series
    \end{itemize}
    
    \item \textbf{Analysis \& Signal Development}
    \begin{itemize}
        \item Perform correlation regime analysis
        \item Develop asset-class specific signals
        \item Test signal stability and turnover
        \item Validate economic intuition
    \end{itemize}
    
    \item \textbf{Portfolio Construction}
    \begin{itemize}
        \item Implement hierarchical risk parity
        \item Apply risk budgeting overlay
        \item Develop rebalancing framework
        \item Create monitoring dashboard
    \end{itemize}
    
    \item \textbf{Risk Management}
    \begin{itemize}
        \item Design stress testing framework
        \item Implement risk limits
        \item Create performance attribution
        \item Monitor factor exposures
    \end{itemize}
\end{enumerate}

\end{document}
