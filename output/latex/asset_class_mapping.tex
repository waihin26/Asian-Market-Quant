
\documentclass{article}
\usepackage[margin=1in]{geometry}
\usepackage{tabularx}
\usepackage{booktabs}
\usepackage{color}
\usepackage{graphicx}

\title{Asset Class Mapping for Asian Market Quant Project}
\author{Wong Wai Hin}
\date{\today}

\begin{document}

\maketitle

\section{Asset Class Mapping}

This document outlines the asset class categorization for our cross-asset Asian markets project. 
We have categorized the tickers into five main asset classes for systematic analysis and risk budgeting.


\begin{table}[h!]
\centering
\caption{Asset Class Mapping for Asian Markets}
\label{tab:asset_class_mapping}
\begin{tabularx}{\textwidth}{|l|X|l|X|}
\hline
\textbf{Ticker Range} & \textbf{Asset Class} & \textbf{Currency} & \textbf{Comment} \\
\hline
MXAP \ldots FMETF & Emerging-Asia equity indices & Mostly USD & Regional beta + macro sensitivity \\
\hline
GOLDS \ldots S & Commodities (Gold spot, Brent front-month, generic Softs, Philippines gold ETF) & USD & Adds inflation hedge, carry via roll \\
\hline
SPX \ldots NKY & Developed-market equity benchmarks & USD / JPY & Good stress-test proxies \\
\hline
USDPHP \ldots USDJPY & FX & USD notional & Carry + momentum rich \\
\hline
USGG5YR \ldots GTUSDPH5Y & Sovereign & USD & Duration + EM credit risk \\
\hline
\end{tabularx}
\end{table}

\section{Risk Budgeting Framework}

Based on our asset class mapping, we propose the following risk budget allocation for portfolio construction.
This allocation will be used in the hierarchical risk parity (HRP) overlay.


\begin{table}[h!]
\centering
\caption{Risk Budget Allocation}
\label{tab:risk_budget}
\begin{tabular}{|l|c|}
\hline
\textbf{Asset Class} & \textbf{Allocation (\%)} \\
\hline
Equities & 60.0\% \\
\hline
Rates & 20.0\% \\
\hline
Fx & 10.0\% \\
\hline
Commodities & 10.0\% \\
\hline
\end{tabular}
\end{table}

\section{Asset Class Descriptions}

\subsection{Emerging-Asia Equity}
This category includes the major Asian equity indices (MXAP, MXAS) and country-specific indices 
(PCOMP for Philippines, JCI for Indonesia, etc.). These provide exposure to regional beta with 
varying degrees of macro sensitivity.

\subsection{Commodities}
Our commodity exposure includes gold spot (GOLDS), Brent crude front-month (CO1), generic Softs (S 1), 
and a Philippines gold ETF (FMETF). This basket provides inflation hedging properties and potential
carry from rolling futures contracts.

\subsection{Developed-Market Equity}
We include S\&P 500 (SPX) and Nikkei 225 (NKY) as developed market benchmarks that serve as 
useful proxies for stress testing our portfolio and measuring correlation regimes.

\subsection{FX Crosses}
We track several USD crosses including USDPHP, USDMYR, USDIDR, USDSGD, and USDJPY. 
These provide exposure to carry and momentum factors that tend to work well in Asia.

\subsection{Sovereign Yields}
Our rates exposure includes 5-year sovereign yields: US Treasury (USGG5YR), Philippine government bonds (GTPHP5yr), 
and USD-denominated Philippine sovereign debt (GTUSDPH5Y). These provide duration exposure and EM credit risk.

\section{Next Steps}

With this asset class mapping complete, we will proceed to:

\begin{enumerate}
    \item Implement data cleaning and currency normalization
    \item Perform exploratory analysis to identify correlations and regime changes
    \item Design signal prototypes for each asset class
    \item Apply hierarchical risk parity within our risk budget framework
\end{enumerate}

\end{document}
